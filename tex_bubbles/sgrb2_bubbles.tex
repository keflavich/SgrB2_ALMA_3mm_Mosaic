
Title: Evidence for supernova and shock-driven heating in the Sgr B2 cloud

New ALMA data reveal a network of bubbles and cavities excavated in the
molecular gas within Sgr B2.  These bubbles provide direct evidence for
the feedback from the forming massive clusters.


- HC3N 10-9 lines show sharp edge features that trace part of a large bow also
  seen in single dish 13CO 2-1 images

  - This large bow has no associated free-free emission
    TODO: What are the upper limits from the 20cm VLA maps?

  - The interior shows no sign of free-free or synchrotron emission

  - The overall geometry indicates an outflow cone toward us along the line of
    sight

  - HC3N in the shocks are consistent with warm temperatures but *not* with
    IR-pumped temperatures like in Sgr B2 N / M
    -no vibrationally excited HC3N is detected [upper limit?]
    -HC3N 24-23 is exceedingly faint or entirely absent

Energetics:
- HC3N abundance ~10^-9 based on, e.g., Gwenlan2000.
- Temperature ~100-150 K based on CH3CN (try to do better)
- Expansion velocity 30 km/s?



\section{A streamer of forming stars in the south of Sgr B2}
The most substantial new feature of the ALMA data is an extensive streamer of
unresolved sources along the south portion of the mapped area.  A narrow
streamer extends from Sgr B2 M south for $\sim8$ pc.  The stream is well-traced
by a filament of molecular emission in \methylcyanide (and...).

The presence of star formation exclusively along this filament (and in the core
clusters) demonstrates that HCN, HNC, and HCO+ are extraordinarily poor tracers
of star formation activity: they are \emph{anti}-correlated with the actual
zones of star formation.  The overall galaxy-scale correlations between
HCN/HNC/HCO+ and star formation likely mark the presence of collapsing giant
molecular clouds, but these emission lines are a step separated from the actual
formation of stars.

% does star formation in this streamer correspond to any sort of threshold?

The great extent of this streamer suggests that these stars are not forming a
cluster.  There is clearly an unclustered mode of star formation possible in
the Galactic center.
