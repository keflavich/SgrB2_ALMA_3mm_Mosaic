\newcommand{\ncores}{138\xspace}
\begin{document}


\section{Analysis}

\subsection{Continuum Source Identification}
We selected continuum point sources as candidate cores or protostars by eye.
An automated selection is not viable across the majority of the field because
there are many extended \hii regions that dominate the overall map emission.  A
future automated selection algorithm may work if images at comparable
resolution at other frequencies become availabe; the \hii-region sources could
then be excluded.  Additionally, however, there are substantial imaging
artifacts produced by the extremely bright emission sources in Sgr B2 M ($S_{3
mm,max} > 0.8$ Jy) and Sgr B2 N ($S_{3 mm,max} > 0.3$ Jy) that make automated
source identification particularly challenging in the regions they are most
common.

\section{Results}
We detected \ncores compact continuum sources.  The majority of these are
likely to be dust-dominated protostellar envelopes, though many are free-free
dominated hypercompact \hii regions.  They are unlikely to be dusty prestellar
cores, since they are predominantly unresolved or barely resolved, with
$R<1\arcsec$ ($R<8500$ AU).


Possible approaches:
-Create a column density map from one or more of the Herschel / Laboca / Bolocam / SCUBA maps,
then make some cumulative N(cores) vs column plots.  Assess SF thresholds.

\end{document}
