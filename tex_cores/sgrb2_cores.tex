\documentclass{emulateapj}
\input{preface}
\newcommand{\ncores}{138\xspace}

\begin{document}
\title{Sgr B2 ALMA}
\titlerunning{Sgr B2 ALMA}
\authorrunning{Ginsburg et al}
% for future reference, this is probably a better approach:
% https://github.com/dfm/peerless/blob/af483ced97045c213650ed807c430b2f87d2c8c9/document/ms.tex#L104
% assuming it's compatible with A&A
%\newcommand{\nrao}{$^{1}$}
%\newcommand{\eso}{$^{2}$}
\newcommand{\nraojansky}{\affiliation{\it{Jansky fellow of the National Radio Astronomy Observatory, 1003 Lopezville Rd, Socorro, NM 87801 USA }}}
\newcommand{\nrao}{\affiliation{\it{National Radio Astronomy Observatory, 1003 Lopezville Rd, Socorro, NM 87801 USA }}}
\newcommand{\nraocv}{\affiliation{\it{National Radio Astronomy Observatory, 520 Edgemont Rd, Charlottesville, VA 22903, USA }}}
\newcommand{\eso}{ \affiliation{\it{ European Southern Observatory, Karl-Schwarzschild-Stra{\ss}e 2, D-85748 Garching bei München, Germany } } }


\newcommand{\radboud}{\affiliation{\it{Department of Astrophysics/IMAPP, Radboud University Nijmegen, PO Box 9010, 6500 GL Nijmegen, the Netherlands}}}
\newcommand{\allegro}{\affiliation{\it{ALLEGRO/Leiden Observatory, Leiden University, PO Box 9513, 2300 RA Leiden, the Netherlands}}}
\newcommand{\zah}{\affiliation{\it{Astronomisches Rechen-Institut, Zentrum f{\"u}r Astronomie der Universit{\"a}t Heidelberg, M{\"o}nchhofstra{\ss}e 12-14, 69120 Heidelberg, Germany}}}
\newcommand{\casa}{\affiliation{\it{CASA, University of Colorado, 389-UCB, Boulder, CO 80309}} }
\newcommand{\jodrell}{\affiliation{\it{Jodrell Bank Centre for Astrophysics, School of Physics and Astronomy, University of Manchester, Oxford Road, Manchester M13 9PL, UK}}}
\newcommand{\morelia}{\affiliation{\it{Instituto de Radioastronom{\'i}a y Astrof{\'i}sica, UNAM, A.P. 3-72, Xangari, Morelia, 58089, Mexico}}}
\newcommand{\sjsu}{\affiliation{\it{{San Jose State University, One Washington Square, San Jose, CA 95192}}}}
\newcommand{\herts}{\affiliation{\it{Centre for Astrophysics Research, University of Hertfordshire, College Lane, Hatfield, AL10 9AB, UK}}}
\newcommand{\uofa}{\affiliation{\it{Dept. of Physics, University of Alberta, Edmonton, Alberta, Canada}}}
\newcommand{\arcetri}{\affiliation{\it{INAF-Osservatorio Astrofisico di Arcetri, Largo E. Fermi 5, I-50125, Florence, Italy } } }
\newcommand{\exclus}{\affiliation{\it{Excellence Cluster Universe, Boltzman str. 2, D-85748 Garching bei M\"unchen, Germany } }}
\newcommand{\ljmu}{\affiliation{\it{Astrophysics Research Institute, Liverpool John Moores University, 146 Brownlow Hill, Liverpool L3 5RF, UK }}}
\newcommand{\koeln}{\affiliation{\it{Cologne}}}
\newcommand{\mpia}{\affiliation{\it{Max-Planck-Institute for Astronomy, Koenigstuhl 17, 69117 Heidelberg, Germany}}}
\newcommand{\agnesscott}{\affiliation{\it{Agnes Scott College, 141 E. College Ave., Decatur, GA 30030}}}
\newcommand{\chile}{\affiliation{\it{Universidad de Chile}}}
\newcommand{\leiden}{\affiliation{\it{Leiden Observatory, Leiden University, PO Box 9513, NL-2300 RA Leiden, the Netherlands }}}
\newcommand{\mpe}{\affiliation{\it{MPE Garching}}}
\newcommand{\boston}{\affiliation{\it{Boston}}}
\newcommand{\cfa}{\affiliation{\it{Harvard-Smithsonian Center for Astrophysics, 60 Garden St. Cambridge, MA 02138}}}
\newcommand{\usf}{\affiliation{\it{University of South Florida, Physics Department, 4202 East Fowler Ave, ISA 2019 Tampa, FL 33620}}}
\newcommand{\uconn}{\affiliation{\it{University of Connecticut, Department of Physics, 2152 Hillside Rd., Storrs, CT 06269}}}

\author[0000-0001-6431-9633]{Adam Ginsburg}
\nraojansky
\eso

%\author{
%Adam Ginsburg{\nrao},
%\begin{flushleft}
%\institutions
%\end{flushleft}
%        }
%
%\institute{
%    {\nrao}{\it{National Radio Astronomy Observatory, Socorro, NM 87801 USA\\
%                      \email{aginsbur@nrao.edu} 
%                      }} \\
%    }
%
\correspondingauthor{Adam Ginsburg}
\email{aginsbur@nrao.edu; adam.g.ginsburg@gmail.com}



\author{Chris De~Pree}
\agnesscott


\author{Elisabeth A.C. Mills }
\boston

\author{John Bally}
\casa
\author{Jeremy Darling}
\casa

\author{Cara Battersby}
\cfa
\uconn
\author{David Wilner}
\cfa

\author{Guido Garay}
\chile

\author{Fanyi Meng}
\koeln
\author{Alvaro Sanchez-Mong{\'e}}
\koeln
\author{Peter Schilke}
\koeln
\author{Anika Schmiedeke}
\koeln
\mpe

\author{Ashley Barnes}
\ljmu
\author{Nate Bastian }
\ljmu
\author{Steven Longmore}
\ljmu
\author{Daniel Walker}
\ljmu


\author{Yanett Contreras}
\leiden

\author{Henrik Beuther}
\mpia
\author{Jonathan Henshaw}
\mpia

\author{Jaime Pineda}
\mpe

\author{Roberto Galv{\'a}n-Madrid}
\morelia

\author{Crystal Brogan}
\nraocv
\author{Joanna Corby}
\nraocv
\usf
\author{Juergen Ott}
\nrao
\author{Todd Hunter}
\nraocv




\author{J.~M.~Diederik Kruijssen}
\zah


\begin{abstract}
    Things
\end{abstract}

\ifpdf
\maketitle
\fi



\section{Observation and Data Reduction}
Data were acquired as part of ALMA project 2013.1.00269.S.  Observations were
taken with the 12m Total Power array, the ALMA 7m array, and in two
configurations with the ALMA 12m array.  The setup included the maximum allowed
number of channels, 15360, across 4 spectral windows in a single polarization;
the single-polarization mode was adopted to support moderate spectral resolution
across the broad bandwidth.

The ALMA QA2 calibrated measurement sets were combined to make a single
high-resolution, high-dynamic range data set.  We imaged the continuum jointly
across all four bands, and found that the central regions surrounding Sgr B2M
were severely affected by artifacts that could not be cleaned out.  We
therefore ran 3 iterations of phase-only self-calibration and one iteration of
amplitude + phase self-calibration to yield a substantially improved image.
The total dynamic range, measured as the peak brightness in Sgr B2 to the RMS
noise in a signal-free region of the image, is 22000 (noise $\sim0.08$
mJy/beam), while the dynamic range within one primary beam ($\sim0.5$\arcmin)
of Sgr B2M is only 3700 (noise $\sim0.5$ mJy/beam).  Because of the dynamic
range limitations, and an empirical determination that clean did not converge
if allowed to go too deep, we cleaned to a threshold of 0.5 mJy/beam across the
image.

We also produced cubes of all of the spectral lines.  These were lightly cleaned
with only 200 iterations of cleaning.  No self-calibration was applied.

\section{Analysis}

\subsection{Continuum Source Identification}
We selected continuum point sources as candidate cores or protostars by eye.
An automated selection is not viable across the majority of the field because
there are many extended \hii regions that dominate the overall map emission.  A
future automated selection algorithm may work if images at comparable
resolution at other frequencies become availabe; the \hii-region sources could
then be excluded.  Additionally, however, there are substantial imaging
artifacts produced by the extremely bright emission sources in Sgr B2 M ($S_{3
mm,max} > 0.8$ Jy) and Sgr B2 N ($S_{3 mm,max} > 0.3$ Jy) that make automated
source identification particularly challenging in the regions they are most
common.

\section{Results}
We detected \ncores compact continuum sources.  The majority of these are
likely to be dust-dominated protostellar envelopes, though many are free-free
dominated hypercompact \hii regions.  They are unlikely to be dusty prestellar
cores, since they are predominantly unresolved or barely resolved, with
$R<1\arcsec$ ($R<8500$ AU).

\subsection{An examination of star formation thresholds}
Lada et al, and others, have proposed that star formation can only occur above
a certain density or column density threshold\footnote{Column density is more
commonly used because of its observational convenience, but it is physically
meaningless unless high column density leads to high optical depths and thereby
changes the gas's ability to cool.}.  In G0.253+0.016, very little star formation
has been observed \citep{Longmore2013a,Johnston2014a,Rathborne2015a} despite
most of the cloud existing above the locally measured column density threshold.

Since we have detected substantial ongoing star formation in the form of
high-mass protostars and/or protostellar cores, we can assess where these stars
form and whether the same (lack of) a threshold exists in Sgr B2.  We therefore
plot the column density distribution (flux distribution?) and overlay the
cumulative distribution function of the background brightness around the cores.

Comparing Sgr B2 to G0.253, the majority of the Sgr B2 cloud is brighter and at
higher column than G0.253.  The presence of star formation in Sgr B2 nearly all
occurs at a higher column than exists within G0.253 (Figure \ref{fig:fluxhist}).  The lack of SF in the brick
is therefore consistent with the active SF in Sgr B2 and the CMZ's higher SF threshold
is confirmed.

\Figure{figures/flux_histograms_with_core_location_CDF.png}
{Histograms of the brightness measured with a variety of instruments at
different submillimeter bands with the cumulative distribution function (CDF)
of the \emph{background} brightness surrounding each core superposed.  The
X-axis units are arbitrary (because right now I don't know the units of all of
these) except for column, which is in units of cm$^{-2}$ of \hh as derived from
SED fits to Herschel data (Battersby+).  The grey line is of the observed
region in Sgr B2 and the blue line is of G0.253+0.016.  The thick grey line is
the CDF of core background brightness, and is labeled by the right axis.}
{fig:fluxhist}{1}{20cm}

\subsection{TODO}
Possible approaches:
-Create a column density map from one or more of the Herschel / Laboca / Bolocam / SCUBA maps,
then make some cumulative N(cores) vs column plots.  Assess SF thresholds.


\input{solobib}

\end{document}
