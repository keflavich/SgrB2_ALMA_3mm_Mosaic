\documentclass{emulateapj}
\input{preface}
\newcommand{\ncores}{138\xspace}

\begin{document}
\title{A catalog of 3mm point sources in the Sgr B2 cloud: signs of extended star formation in a CMZ cloud}
\titlerunning{Sgr B2 ALMA}
\authorrunning{Ginsburg et al}
% for future reference, this is probably a better approach:
% https://github.com/dfm/peerless/blob/af483ced97045c213650ed807c430b2f87d2c8c9/document/ms.tex#L104
% assuming it's compatible with A&A
%\newcommand{\nrao}{$^{1}$}
%\newcommand{\eso}{$^{2}$}
\newcommand{\nraojansky}{\affiliation{\it{Jansky fellow of the National Radio Astronomy Observatory, 1003 Lopezville Rd, Socorro, NM 87801 USA }}}
\newcommand{\nrao}{\affiliation{\it{National Radio Astronomy Observatory, 1003 Lopezville Rd, Socorro, NM 87801 USA }}}
\newcommand{\nraocv}{\affiliation{\it{National Radio Astronomy Observatory, 520 Edgemont Rd, Charlottesville, VA 22903, USA }}}
\newcommand{\eso}{ \affiliation{\it{ European Southern Observatory, Karl-Schwarzschild-Stra{\ss}e 2, D-85748 Garching bei München, Germany } } }


\newcommand{\radboud}{\affiliation{\it{Department of Astrophysics/IMAPP, Radboud University Nijmegen, PO Box 9010, 6500 GL Nijmegen, the Netherlands}}}
\newcommand{\allegro}{\affiliation{\it{ALLEGRO/Leiden Observatory, Leiden University, PO Box 9513, 2300 RA Leiden, the Netherlands}}}
\newcommand{\zah}{\affiliation{\it{Astronomisches Rechen-Institut, Zentrum f{\"u}r Astronomie der Universit{\"a}t Heidelberg, M{\"o}nchhofstra{\ss}e 12-14, 69120 Heidelberg, Germany}}}
\newcommand{\casa}{\affiliation{\it{CASA, University of Colorado, 389-UCB, Boulder, CO 80309}} }
\newcommand{\jodrell}{\affiliation{\it{Jodrell Bank Centre for Astrophysics, School of Physics and Astronomy, University of Manchester, Oxford Road, Manchester M13 9PL, UK}}}
\newcommand{\morelia}{\affiliation{\it{Instituto de Radioastronom{\'i}a y Astrof{\'i}sica, UNAM, A.P. 3-72, Xangari, Morelia, 58089, Mexico}}}
\newcommand{\sjsu}{\affiliation{\it{{San Jose State University, One Washington Square, San Jose, CA 95192}}}}
\newcommand{\herts}{\affiliation{\it{Centre for Astrophysics Research, University of Hertfordshire, College Lane, Hatfield, AL10 9AB, UK}}}
\newcommand{\uofa}{\affiliation{\it{Dept. of Physics, University of Alberta, Edmonton, Alberta, Canada}}}
\newcommand{\arcetri}{\affiliation{\it{INAF-Osservatorio Astrofisico di Arcetri, Largo E. Fermi 5, I-50125, Florence, Italy } } }
\newcommand{\exclus}{\affiliation{\it{Excellence Cluster Universe, Boltzman str. 2, D-85748 Garching bei M\"unchen, Germany } }}
\newcommand{\ljmu}{\affiliation{\it{Astrophysics Research Institute, Liverpool John Moores University, 146 Brownlow Hill, Liverpool L3 5RF, UK }}}
\newcommand{\koeln}{\affiliation{\it{Cologne}}}
\newcommand{\mpia}{\affiliation{\it{Max-Planck-Institute for Astronomy, Koenigstuhl 17, 69117 Heidelberg, Germany}}}
\newcommand{\agnesscott}{\affiliation{\it{Agnes Scott College, 141 E. College Ave., Decatur, GA 30030}}}
\newcommand{\chile}{\affiliation{\it{Universidad de Chile}}}
\newcommand{\leiden}{\affiliation{\it{Leiden Observatory, Leiden University, PO Box 9513, NL-2300 RA Leiden, the Netherlands }}}
\newcommand{\mpe}{\affiliation{\it{MPE Garching}}}
\newcommand{\boston}{\affiliation{\it{Boston}}}
\newcommand{\cfa}{\affiliation{\it{Harvard-Smithsonian Center for Astrophysics, 60 Garden St. Cambridge, MA 02138}}}
\newcommand{\usf}{\affiliation{\it{University of South Florida, Physics Department, 4202 East Fowler Ave, ISA 2019 Tampa, FL 33620}}}
\newcommand{\uconn}{\affiliation{\it{University of Connecticut, Department of Physics, 2152 Hillside Rd., Storrs, CT 06269}}}

\author[0000-0001-6431-9633]{Adam Ginsburg}
\nraojansky
\eso

%\author{
%Adam Ginsburg{\nrao},
%\begin{flushleft}
%\institutions
%\end{flushleft}
%        }
%
%\institute{
%    {\nrao}{\it{National Radio Astronomy Observatory, Socorro, NM 87801 USA\\
%                      \email{aginsbur@nrao.edu} 
%                      }} \\
%    }
%
\correspondingauthor{Adam Ginsburg}
\email{aginsbur@nrao.edu; adam.g.ginsburg@gmail.com}



\author{Chris De~Pree}
\agnesscott


\author{Elisabeth A.C. Mills }
\boston

\author{John Bally}
\casa
\author{Jeremy Darling}
\casa

\author{Cara Battersby}
\cfa
\uconn
\author{David Wilner}
\cfa

\author{Guido Garay}
\chile

\author{Fanyi Meng}
\koeln
\author{Alvaro Sanchez-Mong{\'e}}
\koeln
\author{Peter Schilke}
\koeln
\author{Anika Schmiedeke}
\koeln
\mpe

\author{Ashley Barnes}
\ljmu
\author{Nate Bastian }
\ljmu
\author{Steven Longmore}
\ljmu
\author{Daniel Walker}
\ljmu


\author{Yanett Contreras}
\leiden

\author{Henrik Beuther}
\mpia
\author{Jonathan Henshaw}
\mpia

\author{Jaime Pineda}
\mpe

\author{Roberto Galv{\'a}n-Madrid}
\morelia

\author{Crystal Brogan}
\nraocv
\author{Joanna Corby}
\nraocv
\usf
\author{Juergen Ott}
\nrao
\author{Todd Hunter}
\nraocv




\author{J.~M.~Diederik Kruijssen}
\zah


\begin{abstract}
We report the detection of $>100$ sources in the Sgr B2 clouds with extents
smaller than 5000 AU.  These sources are most likely to be protostars or
centrally condensed prestellar cores.  The spatial distribution of these sources
demonstrates that Sgr B2 is experiencing a highly extended star formation
event, not just an isolated `starburst' within the protocluster regions M, N,
and S.
\end{abstract}

\ifpdf
\maketitle
\fi



\section{Observation and Data Reduction}
Data were acquired as part of ALMA project 2013.1.00269.S.  Observations were
taken with the 12m Total Power array, the ALMA 7m array, and in two
configurations with the ALMA 12m array.  The setup included the maximum allowed
number of channels, 30720, across 4 spectral windows in a single polarization;
the single-polarization mode was adopted to support moderate spectral resolution
across the broad bandwidth.

The ALMA QA2 calibrated measurement sets were combined to make a single
high-resolution, high-dynamic range data set.  We imaged the continuum jointly
across all four bands, and found that the central regions surrounding Sgr B2M
were severely affected by artifacts that could not be cleaned out.  We
therefore ran 3 iterations of phase-only self-calibration and one iteration of
amplitude + phase self-calibration to yield a substantially improved image.
The total dynamic range, measured as the peak brightness in Sgr B2 to the RMS
noise in a signal-free region of the image, is 22000 (noise $\sim0.08$
mJy/beam), while the dynamic range within one primary beam ($\sim0.5$\arcmin)
of Sgr B2M is only 3700 (noise $\sim0.5$ mJy/beam).  Because of the dynamic
range limitations, and an empirical determination that clean did not converge
if allowed to go too deep, we cleaned to a threshold of 0.5 mJy/beam across the
image.  We performed this same process for both the longest-baseline data only
(resolution $\sim0.5\arcsec$, largest angular scale theoretically 15\arcsec\
[the shortest baseline] but more practically $\sim7$\arcsec\ [the 5th percentile
baseline length]) and the merged 7m + two 12m configuration data.  The merged
data are more useful for studying extended structures but have lower dynamic
range, while the long-baseline-only data are excellent for extracting and
analyzing pointlike or compact sources.

We also produced cubes of all of the spectral lines.  These were lightly
cleaned with only 200 iterations of cleaning.  No self-calibration was applied.
Before continuum subtraction, dynamic range related artifacts similar to those
in the continuum images were present, but these structures are identical across
frequencies, and were therefore removable in the image domain.  We use
median-subtracted cubes for the majority of our analysis, noting that the only
location in which an error on the continuum $>5\%$ is expected is the Sgr B2
North core \citep{Sanchez-Monge2017a}.

\section{Analysis}

\subsection{Continuum Source Identification}
We selected continuum point sources as candidate cores or protostars by eye.
An automated selection is not viable across the majority of the field because
there are many extended \hii regions that dominate the overall map emission.  A
future automated selection algorithm may work if images at comparable
resolution at other frequencies become availabe; the \hii-region sources could
then be excluded.  Additionally, however, there are substantial imaging
artifacts produced by the extremely bright emission sources in Sgr B2 M ($S_{3
mm,max} > 0.8$ Jy) and Sgr B2 N ($S_{3 mm,max} > 0.3$ Jy) that make automated
source identification particularly challenging in the regions they are most
common.

\section{Results}

We detected \ncores compact continuum sources.  The majority of these are
likely to be dust-dominated protostellar envelopes, though some are free-free
dominated hypercompact \hii regions.  They are unlikely to be dusty prestellar
cores, since they are predominantly unresolved or barely resolved, with
$R<1\arcsec$ ($R<8500$ AU).  Instead, most of these sources are likely to be
protostellar cores with a central heating source and a dusty envelope.

\subsection{Source Classification}

We first note some key properties of dust at 3 mm.   At 8.4 kpc, a 1 mJy source
corresponds to an optically thin dust mass of $M(40\mathrm{K})=18$ \msun or
$M(20\mathrm{K})=38$ \msun assuming a dust opacity index $\beta=1.75$ to
extrapolate the \citet{Ossenkopf1994a} opacity to $\kappa_{3mm}=0.0018$ cm$^2$
g$^{-1}$.  Our dust-only 5-$\sigma$ sensitivity limit at 40 K therefore ranges
from $M>7$ \msun to $M>45$ \msun across the map.  If we were to assume that
these are all cold, dusty sources, as is typically (and reasonably) assumed for
local clouds, they would be extremely massive and dense, with the lowest
measurable density being $n(40\mathrm{K}) > 3\ee{6}$ \percc (corresponding to 7
\msun in a 0.5\arcsec radius sphere).  Such extremes objects are possible, but
since we have detected $>100$ of these sources, it makes sense to evaluate
other possibilities.



Some validation of the protostellar hypothesis comes from catalog matching.
The \citet{Caswell2010a} Methanol Multibeam Survey identified 11 sources in our
observed field of view, of which 10 have a clear match in our catalog.
Others of the sources have been identified with known \hii regions from
\citet{Gaume1995a}.


We compare our detected sample to that of the Herschel Orion Protostar Survey
\citep[HOPS;][]{Furlan2016a} in order to get a general sense of what types of
sources we have detected.  Figure \ref{fig:hopshist} shows the HOPS source
fluxes at 870\um scaled to 3 mm assuming a dust opacity index $\beta=1.5$,
which is shallower than usually inferred.  The 870\um data were acquired with a
$\sim20\arcsec$ FWHM beam, which translates to a resolution $\sim1\arcsec$ at
8.4 kpc, so our beam size is very similar to theirs.  The HOPS sources are all
fainter than the Sgr B2 sources.  The brightest HOPS source would only be 0.2
mJy in Sgr B2, or about a 4-$\sigma$ source - below our detection threshold
even in the noise-free regions of the map.  We can therefore conclude that
the Sgr B2 sources are much more luminous and therefore massive protostars.

% Bontemps+ 2010: 3.5mm flux of N63 ~ 36 mJy, -> 1 mJy @ Sgr B2

\Figure{figures/core_peak_intensity_histogram_withHOPS.png}
{A histogram combining the detected Sgr B2 cores with predicted flux densities
based on the HOPS \citep{Furlan2016a} survey.  The HOPS histogram shows the 870
\um data from that survey scaled to 3 mm assuming $\beta=1.5$.  Every HOPS
source is well below the detection threshold for our observations.}
{fig:hopshist}{1}{15cm}

We could assume that the stellar mass is linearly proportional to the 3 mm
continuum flux density to infer the instantaneous stellar fraction in the cloud.
However, this assumption is immediately at odds with our expectations: the
flux density distribution follows a powerlaw with slope $\alpha=1.94\pm0.07$
\citep[fitted with the MLE method of][]{Clauset2007a}, far shallower than the
$\alpha\sim2.35$ expected for a normal IMF.

\Figure{figures/core_peak_fluxdensity_powerlawfit.png}
{A histogram of the peak flux density of the observed sources excluding known
\hii regions with a powerlaw fit shown.  The fitted powerlaw is an excellent
fit to the data, but is far shallower than the IMF slope, with
$\alpha=1.94\pm0.07$.  The two brightest regions are Sgr B2M f1 and Sgr B2N K2,
which may be dominated by free-free emission but likely also contain a large
dust mass.}
{fig:fluxhist}{1}{15cm}

If we make the very simplistic assumptions that the sources we detect are all
$L\gtrsim2000$ \Lsun ($M\gtrsim8 \msun$),  we can infer the total stellar mass.
Using a \citet{Kroupa2001a} mass function with $M_{max}=200$ \msun, 0.45\% of
the mass is contained in $M>8\msun$ stars.  Using $M=8\msun$ as the lower-limit
case for each source, the identified sources have total mass $M(>8)=1800\msun$.
The total stellar mass implied is $M_{tot} = 3.9\ee{5}$ \msun.  If we use the
mean stellar mass for $M>8$ \msun, $\bar{M}=21.1$ \msun, then $M_{tot}=1\ee{6}$
\msun.





% While we identified \ncores sources from the continuum data, since we have only
% a single continuum band available, it is difficult to classify most of these
% except to say that they are certainly forming or recently formed stars.
% However, for a small subset, we have spectral line detections in either
% molecular or ionized species that tell us qualitatively whether a source is
% ionizing an \hii region or is surrounded by interesting molecular species.
% 
% \todo{Continue here - give the subset of sources with good line IDs (which is
% probably a by-hand process) and show example spectra of something not Sgr B2 M
% or N}

\subsection{An examination of star formation thresholds}
Lada et al, and others, have proposed that star formation can only occur above
a certain density or column density threshold\footnote{Column density is more
commonly used because of its observational convenience, but it is physically
meaningless unless high column density leads to high optical depths and thereby
changes the gas's ability to cool.}.  In G0.253+0.016, very little star formation
has been observed \citep{Longmore2013a,Johnston2014a,Rathborne2015a} despite
most of the cloud existing above the locally measured column density threshold.

Since we have detected substantial ongoing star formation in the form of
high-mass protostars and/or protostellar cores, we can assess where these stars
form and whether the same (lack of) a threshold exists in Sgr B2.  We therefore
plot the column density distribution (flux distribution?) and overlay the
cumulative distribution function of the background brightness around the cores.

Comparing Sgr B2 to G0.253, the majority of the Sgr B2 cloud is brighter and at
higher column than G0.253.  The presence of star formation in Sgr B2 nearly all
occurs at a higher column than exists within G0.253 (Figure
\ref{fig:fluxhist}).  The lack of SF in the brick is therefore consistent with
the active SF in Sgr B2 and the CMZ's higher SF threshold is confirmed.

\subsubsection{Comparison to Lada, Lombardi, and Alves 2010}
In this section, we compare the star formation threshold in Sgr B2 to that in
local clouds performed by \citet{Lada2010a}.  They determined that all star
formation in local clouds occurs above a column density threshold $M_{thresh} >
116$ \msun pc$^{-2}$, or $N_{thresh}(\hh) > 5.2\ee{21}$ \persc assuming the
mean particle mass is 2.8 amu \citep{Kauffmann2008a}.  We first note, then,
that \emph{all pixels} in our column density maps are above this threshold
by \emph{at least} a factor of 10.

However, the CMZ is 8.4 kpc away from us in the direction of our Galaxy's
center, meaning there is a potentially enormous amount of material unassociated
with the Sgr B2 cloud along the line of sight.  This material may be as low as
5\ee{21} \persc or as high as 5\ee{22} \persc.  The latter value is
approximately the lowest seen within our field of view and would imply that
there is a perfect vacuum surrounding the dense gas in the Sgr B2 cloud.
Even with the very aggressive foreground value of 5\ee{22} \persc subtracted,
essentially the whole cloud exists above this threshold.  We can therefore
immediately rule out the possibility that there is a universal star formation
column threshold, since a large fraction of the observed volume exhibits
no hint at all of star formation activity.

-What kind of stars are we sensitive to?  Are they?


\Figure{figures/flux_histograms_with_core_location_CDF.png}
{Histograms of the brightness measured with a variety of instruments at
different submillimeter bands with the cumulative distribution function (CDF)
of the \emph{background} brightness surrounding each core superposed.  The
X-axis units are arbitrary (because right now I don't know the units of all of
these) except for column, which is in units of cm$^{-2}$ of \hh as derived from
SED fits to Herschel data (Battersby+).  The grey line is of the observed
region in Sgr B2 and the blue line is of G0.253+0.016.  The thick grey line is
the CDF of core background brightness, and is labeled by the right axis.}
{fig:fluxhist}{1}{20cm}



\input{solobib}

\end{document}
