\documentclass[twocolumn]{aastex61}
\input{preface}
\begin{document}
Sgr B2 Cluster Formation Efficiency

\section{Introduction}

The cluster formation efficiency is an important measurement...
Kruijssen+ 2012

Notes to self:
\begin{itemize}
    \item mass of stars, gas in M, N, distributed population, S? [mostly taken from Schmiedeke] [DONE?]
    \item virial parameters?  Use DePree's data for velocity dispersion?  [added my own H41a; DePree's are useful though] [ DONE - note to Diederik - this is for stars, not gas ]
    \item cluster definition: purely radial.  Is S a cluster?  Is NE?  [assuming "No" below; have added discussion about effect of expanding N]
    \item Age difference.  What (maximal) errors can this impose?
\end{itemize}

\section{Observational Summary}

We use the catalogs described in \citet{Ginsburg2017c} and \citet{De-Pree2015a}.


\section{The mass of the clusters}
Cluster masses are determined by counting the number of HII regions and high-mass
protostellar cores associated with each of the clusters.

In Table 2 of Ginsburg et al 2017, four clusters were considered: N, M, NE, and
S.  Here, we re-evaluate the ``clusters" in NE and S.  These regions are not
centrally concentrated and do not have many sources within the named region.
They are both moderate mass and, at present, do not appear likely to form bound
clusters.  We therefore exclude them from the analysis {\color{red} or, do with/without
them}.

\input{cluster_mass_estimates_cfe.tex}

\subsection{Cluster membership in Sgr B2 M}
\citet{Schmiedeke2016a} marked the Sgr B2 M cluster as a 13\arcsec  (0.5 pc) radius
region centered on Sgr B2 M F3.  Within this volume, there are 47 \hii regions
cataloged by \citet{De-Pree201?} from their 0.05\arcsec resolution 7 mm Q-band
VLA observations.  There are 17 non-\hii-region cores, the faintest of which is
1.3 mJy at 3 mm.  By extrapolating the \hii region counts,
\citet{Ginsburg2017c} inferred a total stellar mass of 1.5\ee{4} \msun.

% ((2.3e24*u.cm**-2) * (40*u.arcsec/2.35 * 8.5*u.kpc)**2 * 2*np.pi * 2.8*u.Da).to(u.Msun, u.dimensionless_angles())
{\color{red} There is some inconsistency in \citet{Schmiedeke2016a}.}
Based on \citet{Schmiedeke2016a}'s column density measurement of $N(\hh)=2.3\ee{24}~\persc$
in a 40\arcsec beam toward Sgr B2 M, the total gas mass in the center-most beam is
$M_{gas, M} = 1.6\ee{5} \msun$.  The instantaneous SFE is only 1\%.

\citet{Schmiedeke2016a} give a gas mass of $M=9.6\ee{3}$ \msun in the Sgr B2 M
cloud in their Table 2, implying an instantaneous SFE of about 60\%.  The total
mass within 0.5 pc is then about $M_M = 2.5\ee{4}$ \msun, and the escape speed
is $v_{esc}=14~\kms$.


% While it is possible that some of these sources are unassociated with Sgr B2,
% their proximity to the Sgr B2 core suggests they are indeed bound...

It is possible the Sgr B2 M cluster is substantially larger, 35\arcsec (1.4 pc).
Within this radius, the `core' count is larger, 52 rather than 17, but the \hii
region count increases only marginally, from 47 to 49.  Since the \hii region-inferred
stellar mass is larger in both cases, we use this estimate, but the presence
of many cores in the outskirts of the Sgr B2 M cluster suggests both that it may
grow in stellar mass by accretion up to $\sim50\%$ and that the lack of cores in the
innermost region is due to incompleteness rather than their absence, as suggested
in \citet{Ginsburg2017c}.

\subsection{Cluster membership in Sgr B2 N}
\citet{Schmiedeke2016a} marked the Sgr B2 N cluster as a 10\arcsec  (0.4 pc) radius circle
centered on Sgr B2 N K2.  \citet{Schmiedeke2016a} identified 3 compact \hii regions
and \citet{Ginsburg2017c} identified 11 cores within this region.  The inferred
total stellar mass is 980-1500 \msun.  However, unlike Sgr B2 M, Sgr B2 N
is gas-dominated, with $M_{gas,N} = 2.8\ee{4}~\msun$ and SFE $\sim5\%$ \citep{Schmiedeke2016a}.
The escape speed from the 0.4 pc cluster is $v_{esc} = 18~\kms$.

Sgr B2 N is therefore better described as a `protocluster', in contrast with
Sgr B2 M, which is a (very) young cluster (YMC).  Sgr B2 N will need to form an
additional several thousand \msun of stars to form a YMC, and will need to do
so at high efficiency.  However, since there is evidence that the protocluster
itself is still rapidly accreting both stars and gas \citep[][cite myself?]{},
this outcome is quite likely.


\subsection{Velocity Dispersion Measurements - boundedness}

We compare our velocity measurements to those of \citet{De-Pree2011a}.
With the new H41$\alpha$ data, we have several new RRL measurements, which increases the velocity dispersion ($\sigma_{1D}$)
from 9 \kms to 12 \kms.  This is significantly lower than the escape velocity.
%1D Velocity Dispersion of 41a: 11.773405291227526, 52a: 9.013305133051228, 66a: 9.52127673916695

However, some individual sources are moving at high velocity with respect to
the average ($\bar{v}_{LSR}(H41\alpha) = 58.5 \kms$,
$\bar{v}_{LSR}(H52\alpha) = 65.8 \kms$), the fastest being G10.47 at
$v_{LSR}=34$ \kms or $v_{center}=24--32$ \kms.  There is a small group
at these highly negative velocities and a projected distance from the center $r<0.1$ pc;
these may be bound to a higher potential than we have inferred above, or they
could be a separate cluster moving along the line of sight.

The \hii region J is separated by 0.4 pc and 16--24 \kms and is a diffuse \hii region.... is it a blowout?   an unconnected star?  hmm.

\subsection{The cluster formation efficiency}
Table \ref{tab:clustermassestimates} shows the breakdown of ongoing star
formation within the Sgr B2 region.  The total inferred mass of recently
formed or forming stars is $M_{*,total}\approx4.6\ee{4}$ \msun spread across
the whole cloud, with $M_{*,clustered}\approx1.7\ee{4}$ \msun concentrated
in the Sgr B2 M and N clusters.  These values imply a \textit{cluster
formation efficiency} $CFE=M_{*,clustered}/M_{*,total} = 37\%$.

We have noted above that the membership of clusters M and N could be expanded,
and while this expansion would have no effect on the estimated mass of the clusters
(because their masses have been inferred from more complete samples of \hii regions),
it would reduce the number of unassociated cores by about 25\%, increasing the inferred
CFE to $\approx43\%$. % this number calculated by hand: m_tot = 46k - (27k*0.25) = 40, 17 / 40 = 43%.
Treating Sgr B2 NE and S as clusters would also serve to increase our CFE estimates,
but only by 2-3\%, so we ignore them.

\input{solobib.tex}

\end{document}
