Sgr B2 Cluster Formation Efficiency

Introduction

The cluster formation efficiency is an important measurement...
Kruijssen+ 2012

- mass of stars, gas in M, N, distributed population, S?
- virial parameters?  Use DePree's data for velocity dispersion?
- cluster definition: purely radial.  Is S a cluster?  Is NE?
- Age difference.  What (maximal) errors can this impose?

Observational Summary

We use the catalogs described in Ginsburg et al 2017, De-Pree et al 2015


\section{The mass of the clusters}
Cluster masses are determined by counting the number of HII regions and high-mass
protostellar cores associated with each of the clusters.

In Table 2 of Ginsburg et al 2017, four clusters were considered: N, M, NE, and
S.  Here, we reconsider the "clusters" in NE and S.  These regions are not
centrally concentrated and do not have many sources within the named region.
They are both moderate mass and, at present, do not appear likely to form bound
clusters.

\subsection{Cluster membership in Sgr B2 M}
\citet{Schmiedeke2016a} marked the Sgr B2 M cluster as a 13\arcsec  (0.5 pc) radius
region centered on Sgr B2 M F3.  Within this volume, there are 47 \hii regions
cataloged by \citet{De-Pree201?} from their 0.05\arcsec resolution 7 mm Q-band
VLA observations.  There are 17 non-\hii-region cores, the faintest of which is
1.3 mJy at 3 mm.  By extrapolating the \hii region counts,
\citet{Ginsburg2017c} inferred a total stellar mass of 15000 \msun.

% ((2.3e24*u.cm**-2) * (40*u.arcsec/2.35 * 8.5*u.kpc)**2 * 2*np.pi * 2.8*u.Da).to(u.Msun, u.dimensionless_angles())
{\color{red} There is some inconsistency in \citet{Schmiedeke2016a}.}
Based on \citet{Schmiedeke2016a}'s column density measurement of $N(\hh)=2.3\ee{24}~\persc$
in a 40\arcsec beam toward Sgr B2 M, the total gas mass in the center-most beam is
$M_{gas, M} = 1.6\ee{5} \msun$.  The instantaneous SFE is only 1\%.

\citet{Schmiedeke2016a} give a gas mass of $M=9.6\ee{3}$ \msun in the Sgr B2
cloud, implying an instantaneous SFE of about 60\%.  The total mass within 0.5
pc is then about $M_M = 2.5\ee{4}$ \msun, and the escape speed is
$v_{esc}=14~\kms$.


% While it is possible that some of these sources are unassociated with Sgr B2,
% their proximity to the Sgr B2 core suggests they are indeed bound...

It is possible the Sgr B2 cluster is substantially larger, 35\arcsec (1.4 pc).
Within this radius, the `core' count is larger, ...

\subsection{Cluster membership in Sgr B2 N}
\citet{Schmiedeke2016a} marked the Sgr B2 N cluster as a 10\arcsec  (0.4 pc) radius
centered on Sgr B2 N K2.  \citet{Schmiedeke2016a} identified 3 compact \hii regions
and \citet{Ginsburg2017c} identified 11 cores within this region.  The inferred
total stellar mass is 980-1500 \msun.  However, unlike Sgr B2 M, Sgr B2 N
is gas-dominated, with $M_{gas,N} = 2.8\ee{4}~\msun$ and SFE $\sim5\%$ \citep{Schmiedeke2016a}.
The escape speed from the 0.4 pc cluster is $v_{esc} = 18~\kms$.


\subsection{Velocity Dispersion Measurements - boundedness}

We compare our velocity measurements to those of \citet{De-Pree2011a}.
1D Velocity Dispersion of 41a: 11.773405291227526, 52a: 9.013305133051228, 66a: 9.52127673916695
